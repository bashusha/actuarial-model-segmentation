\section{Conclusion: Implications for actuarial practice}

Under conditions of limited data and high requirements for decision stability,
actuarial practice requires not only more sophisticated models, but also more
rigorous diagnostic criteria for managing model complexity.

The proposed approach shifts the focus from optimising segmentation to
justifying its necessity. By treating segmentation as a diagnostic hypothesis
rather than as a modelling objective, the actuary gains a framework for making
structural decisions in a disciplined and accountable manner.

It is essential that the absence of segmentation is recognised as a valid and
professionally justified outcome. This perspective improves transparency,
reduces model risk, and facilitates communication of modelling decisions with
stakeholders and regulators.

The method is designed for practical use and can be integrated into existing
actuarial governance processes without reliance on large-scale computational
experiments. This makes it particularly relevant for markets with limited data
and resources, and supports the development of stable and inclusive insurance
systems.
